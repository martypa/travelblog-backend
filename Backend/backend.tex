\documentclass[../main.tex]{subfiles}
\graphicspath{{images/}{../images/}}

\begin{document}
Der Server der Travelblog Applikation erfüllt folgende Aufgaben:
\begin{itemize}
    \item Authentifizierung der User
    \item Authorisierung verschiedener Funktionen
    \item Peristierung von Blogeinträgen
    \item Bereitstellung der Applikation
\end{itemize}
Wie bereits im Kapitel \ref{architektur} beschrieben, kommt als Backend ein NodeJS Platform zum Einsatz mit dem express.js Framework. Weiters wird das REST- Paradigma verwendet um Anfragen vom Client beantworten zu können. Für die Authentifizierung der User kommt ein einfaches User/ Passwort abfrage zum Einsatz. Für die Authorisierung wird dann ein JSON Web Token verwendet. Zur Persistierung der Daten kommt eine MongoDB zum Einsatz.

\subsubsection{Express.js}
Als Serverseitiges Framework kommt express.js zum Einsatz. Express.js wurde ausgwählt da es im Unterricht schon behandelt wurde und es eine gute Integration zum Nodejs bietet. Der Aufbau ist sehr einfach gehalten, da nur eine REST Funktionalität und die Anbindung an die MongoDB Datenbank benötigt wird. Die Backendapplikation besteht aus folgenden Teilen:
\begin{itemize}
    \item bin/www: Einstiegspunkt der Backendapplikation
    \item app.js: Konfigurationen express.js
    \item routes/blog.js: REST Methoden Travelblog
    \item routes/blog.repository.js: MongoDB Methoden
\end{itemize}

\subsubsection{REST-API}
Zur Beschreibung der Kommunikationspunkte im Backend kommt die REST- Architektur zum Einsatz. Wie in \ref{fig:rest-api} ersichtlich, werden vier GET- Methoden und drei POST- Methoden zur Verfügung gestellt. Untenstehend werden die jeweiligen Methoden kurz beschrieben.

\begin{figure}[h]
    \centering
    \includegraphics[width=0.8\textwidth]{rest-api}
    \caption{Verfügbare REST- Methoden}
    \label{fig:rest-api}
\end{figure}
\vspace{0.5cm}

\begin{tabular}{ p{5cm} p{5cm}}
 \hline
 \textbf{Methode} & \textbf{Beschreibung}\\
 \hline
 GET /blog                 &   Gibt alle erstellten Reisen (Travels) zurück                         \\ \hline
 GET /header/:id           &   Gibt den Blogheader der jeweiligen Reise (id = ReiseID) zurück       \\ \hline
 GET /entries/:id          &   Gibt alle Blogeinträge der jeweiligen Reise (id = ReiseID) zurück    \\ \hline
 GET /picture/:pictureName &   Gibt das gewünschte Bild gemäss dem Bildername zurück                \\ \hline
 POST /upload/Picture      &   Können neue Bilder hochgeladen und persistiert werden                \\ \hline
 POST /upload/BlogEntry    &   Können neue Blogeinträge erstellt und persistiert werden             \\ \hline
 POST /upload/createTravel &   Können neue Reisen erstellt werden                                   \\ \hline
\end{tabular}

\subsubsection{Authentifizierung / Authorisierung}


\subsubsection{MongoDB}
Zur Persistierung des Travelblogs kommt die NoSQL Datenbank MongoDB zum Einsatz. Prinzipiell wäre auch ein Einsatz einer SQL Datenbank möglich gewesen, da es sich bei den Blogeinträgen um strukturierte Daten handeln. Der Entscheid für die MongoDB Datenbank ist aufgrund ihrer einfachen Integration in den Nodejs Server und die Möglichkeit kostenlos eine Cloud- Datenbank zu verwenden. Dies bietet den Vorteil, dass man direkt die MongoDB als IaaS (Infrastructe as a Service) verwenden kann.\\
Die Daten werden gesamthaft in einer Datenbank namens "travelblog" gespeichert. Pro Reise (Travel) wird dann innerhalb der Datenbank eine separate Collection erstellt. Diese wird beim Aufruf der REST- Schnittstelle «POST /upload/createTravel» erstellt. Jeder Blogeintrag und der Header wird jeweils in einem eigenen Document gespeichert. Anhand dieses Datenbankmodells können sehr einfache CRUD- Operationen durchgeführt werden. Zugegriffen wird mittels dem MongoDB Client über HTTPS.
\end{document}